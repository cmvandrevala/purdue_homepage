\title{Introductory Electricity, Magnetism, and Optics Practice Problems}
\author{Cyrus Vandrevala}

\documentclass[11pt]{article}
\usepackage{amsmath}
\usepackage[margin=2.5cm]{geometry}
\usepackage[pdftex]{graphicx}

\begin{document}

\maketitle
\tableofcontents
\vspace{50pt}

\subsection*{Useful Constants}
Electron Mass = $9.11 \times 10^{-31}$ kg \\*
Proton Mass = $1.67 \times 10^{-27}$ kg \\*
Elementary Charge = $1.602 \times 10^{-19}$ C \\*
Coulomb's Constant = $8.99 \times 10^9$ Nm$^2$/C$^2$ \\*
Avogadro's Number = $ 6.02 \times 10^{23}$ atoms/mole \\*
Radius of Gold Nucleus = 6.6 fm \\*
Atomic Number of Gold = 79

%%%%%%%%%%%%%%%%%%%%%%%%%%%%%%%%%%%%%%%%%%%%%%%%%%%%%%%%%%%%%%%%%%%%%%%%%%%%%%%%%%%%%%%%%%%%

\pagebreak
\section{Gauss's Law}
\vspace{10pt}

\subsection{True or False \#1}
True or False?  The net electric flux through a closed surface depends on the total charge located outside the surface.\\* \\*
$\Rightarrow$ False

\subsection{True or False \#2}
True or False?  If I enclose a perfect electric dipole within a spherical Gaussian surface, the electric flux at each point on the surface is zero.\\* \\*
$\Rightarrow$ False

\subsection{True or False \#3}
True or False?  Gauss's Law is only true when we are calculating the electric field around objects with spherical or cylindrical symmetry.\\* \\*
$\Rightarrow$ False

\subsection{True or False \#4}
True or False?  If a spherical shell has a positive net flux passing through its surface, then there are no negative charges enclosed within the surface.\\* \\*
$\Rightarrow$ False

\subsection{Offset Hole}
I have a solid conducting sphere of radius R.  I make a small hole inside the sphere and place a charge of -Q inside.  The hole is NOT at the center of the sphere – it is located at a distance of R/2 from the center.  What is the net electric flux passing through the surface of the sphere?\\* \\*
$\Rightarrow -Q/\epsilon_o$

\subsection{Irregularly Shaped Conductor}
Suppose I have a conductor in the shape of an ellipsoid with a net charge of +Q.  What is the orientation of the electric field close to the surface of the conductor?  What is the magnitude of the electric field at the surface of the conductor?\\* \\*
$\Rightarrow +Q/\epsilon_o$, perpendicular to the surface of the conductor

\pagebreak
\subsection{Multiple Choice \#1}
Which of the following statements is NOT true?

\begin{itemize}
 \item[A)] Gauss's Law works for any charge distribution, but it is only convenient to use when there is a high degree of symmetry in the problem
 \item[B)] A Gaussian surface must be a closed surface
 \item[C)] Although it is normally used for point charges, Coulomb's law can be modified to determine the electric field due to a continuous charge distribution
 \item[D)] When I use Gauss's Law, if the charge within my Gaussian surface is zero, then the electric field at every point on the surface must be zero
 \item[E)] If I calculate the electric field from a charge distribution using two different methods (Gauss's law and Coulomb's law for continuous charge distributions), the resulting answers should be identical
\end{itemize}
$\Rightarrow$ D

\subsection{Multiple Choice \#2}
Suppose I enclose a perfect dipole in a spherical Gaussian surface.  Which of the following statements is false?

\begin{itemize}
\item[A)] The net electric flux through the surface is zero
\item[B)] The electric field is zero everywhere on the surface
\item[C)] The electric flux is non-zero at every point on the surface
\item[D)] The net enclosed charge equals zero
\item[E)] The net electric flux would stay the same if I switched the positions of positive and negative charges, keeping everything else constant
\end{itemize}
$\Rightarrow$ B

\subsection{Gauss's Law Calculations}
You must know how to find the electric field from the following charge distributions using Gauss's law.  Please refer to your textbook or class notes for derivations of each of these:

\begin{itemize}
\item Point Charge
\item Infinite Long Thin Wire
\item Infinite Plane
\item Solid Sphere (Conducting and Insulating)
\item Spherical Shell (Conducting and Insulating)
\item Infinitely Long Cylinder (Insulating)
\item Infinitely Long Cylindrical Shell (Insulating)
\end{itemize}

%%%%%%%%%%%%%%%%%%%%%%%%%%%%%%%%%%%%%%%%%%%%%%%%%%%%%%%%%%%%%%%%%%%%%%%%%%%%%%%%%%%%%%%%%%%%

\pagebreak
\section{Electric Potential}
\vspace{10pt}

\subsection{Charging Spheres \#1}
Suppose I have two identical conducting spheres, each of radius R = 1.0 m, each charged to Q = +1.0 C.  I touch a third, uncharged conducting sphere of radius r = 0.5 m to one of the original two spheres and then the other.  What is the final charge on the third sphere with radius r?\\* \\*
$\Rightarrow \frac{4}{9}$ C

\subsection{Charging Spheres \#2}
Suppose I have conducting sphere A with radius $R_A=$ 10 cm and conducting sphere B with radius $R_B=$ 20 cm.  I charge sphere A up to 60 $\mu$C and leave sphere B uncharged.  I then connect the two spheres with a thin wire and let them come to equilibrium.  After I disconnect the wire, what is the final charge on sphere B?  You may assume that none of the charge got “stuck” on the thin wire.\\* \\*
$\Rightarrow 40 \mu$C

\subsection{Charges on a Square \#1}
Four point charges, each with charge -Q, are placed at the corners of a square with side length d.  What is the net electric potential at the center of the square?  Assume V = 0 V at infinite distance.\\* \\*
$\Rightarrow -4 \sqrt{2} \frac{kQ}{d}$

\subsection{Charges on a Square \#2}
Four point charges are placed at the corners of a square with side length d.  Three of these charges are +Q and the last one is -Q.  What is the net electric potential at the center of the square?  Does it matter where the charges are placed?  Assume V = 0 V at infinite distance.\\* \\*
$\Rightarrow 2 \sqrt{2} \frac{kQ}{d}$, It does not matter which charge is at each corner of the square

\subsection{Electric Field to Electric Potential}
The electric field in some region of space is equal to $\vec{E} = 2x$ N/C $\hat{x}$ (i.e. it changes along the x-direction).  Calculate the potential difference between the point (2,0,0) and the point (1,1,1).\\* \\*
$\Rightarrow$ -3.0 V

\subsection{Electric Potential to Electric Field}
Suppose I have an electric potential given by V(x,y,z) = 5x + 3xy – 8z V.  What is the electric field at the point (1,1,0)? \\* \\*
$\Rightarrow -8\hat{x} -3\hat{x} +8\hat{x}$ N/C

\pagebreak
\subsection{Electric Potential to Electric Force}
A region of space has an electric potential given by V = xy + yz + xyz.  I place an electron at the point (2,1,1).  What is the magnitude of the electric force that it feels at that point?\\* \\*
$\Rightarrow 9.86 \times 10^{-19}$ N

%%%%%%%%%%%%%%%%%%%%%%%%%%%%%%%%%%%%%%%%%%%%%%%%%%%%%%%%%%%%%%%%%%%%%%%%%%%%%%%%%%%%%%%%%%%%

\pagebreak
\section{Electric Potential Energy}
\vspace{10pt}

\subsection{Charges on an Equilateral Triangle}
I place three point charges, +Q, +2Q, and -Q, at the corners of an equilateral triangle with side length L.  What is the total electric potential energy of the system?\\* \\*
$\Rightarrow U = -\frac{kQ^2}{L}$

\subsection{Charges on a Square}
I place four charges (+1.0 $\mu$C, -2.0 $\mu$C, +3.0 $\mu$C, and +5.0 $\mu$C) at the corners of a square with side length D = 10.0 m.  How much work would it take to move a fifth charge (q = -7.0 $\mu$C) from infinite distance to the center of the square?  Express your answer in electron volts and joules.\\* \\*
$\Rightarrow$ W = 0.062 J = $3.87 \times 10^{17} eV$

\subsection{Particle Collider (Difficult)}
Suppose you are a scientist working at a particle collider.  You will be shooting two gold nuclei together at high speed so that they will collide and fracture.  In this experiment, you will hold one gold nucleus still and smash the other one into it.  Estimate how much energy you need to fire the second gold nucleus at so that the two nuclei just barely make contact.\\*\\*
$\Rightarrow$ E = 679 MeV\\*
Note: This energy corresponds to a speed of about $2\times 10^7$ m/s when we use E = $\frac{1}{2}mv^2$.  This is just under 10\% the speed of light.  This is a decent approximation to the actual energy needed in this collider.  However, as we approach the speed of light, we need to take relativistic effects into account and this ``classical'' calculation will not work.

%%%%%%%%%%%%%%%%%%%%%%%%%%%%%%%%%%%%%%%%%%%%%%%%%%%%%%%%%%%%%%%%%%%%%%%%%%%%%%%%%%%%%%%%%%%%

\pagebreak
\section{Capacitors}
\vspace{10pt}

\subsection{Box of Capacitors \#1}
Suppose I have a box full of 100 mF capacitors, but I need a 10 mF capacitor for a circuit that I am building.  What combination of 100 mF capacitors would give me what I need?\\* \\*
$\Rightarrow$ 10 capacitors in series

\subsection{Box of Capacitors \#2}
Suppose I have a box full of 30 mF capacitors, but I need a 120 mF capacitor for a circuit that I am building.  What combination of 30 mF capacitors would give me what I need?\\* \\*
$\Rightarrow$ 4 capacitors in parallel

\subsection{Box of Capacitors \#3}
Suppose I have a box full of 10 mF capacitors, but I need a 15 mF capacitor for a circuit that I am building.  What combination of 10 mF capacitors would give me what I need?  Note: There may be multiple solutions to this question.\\* \\*
$\Rightarrow$ One solution: (3 capacitors in parallel) in series with (3 capacitors in parallel)

\subsection{Charge and Energy Stored in a Capacitor \#1}
I hook a capacitor with capacitance C = 10 mF up to a battery with voltage V = 20 kV.  What are the charge and electric potential energy stored in the capacitor?\\* \\*
$\Rightarrow$ Q = 200 C, E = $2 \times 10^6$ J

\subsection{Charge and Energy Stored in a Capacitor \#2}
I hook three capacitors, each with capacitance C = 30 mF, up to a battery with voltage V = 60 V.  All four circuit elements are in series.  What are the charge and electric potential energy stored in each capacitor?\\* \\*
$\Rightarrow$ Q = 0.6 C, E = 6 J (per capacitor)

\subsection{Charge and Energy Stored in a Capacitor \#3}
I hook three capacitors, each with capacitance C = 30 mF, up to a battery with voltage V = 60 V.  All four circuit elements are in parallel.  What are the charge and electric potential energy stored in each capacitor?\\* \\*
$\Rightarrow$ Q = 1.8 C, E = 54 J (per capacitor)

\pagebreak
\subsection{Multiple Choice \#1}
A parallel plate capacitor with initial capacitance C is hooked up to a battery of voltage V.  If I increase the voltage (V), how will the capacitance (C) change?

\begin{itemize}
\item[A)] Increase
\item[B)] Decrease
\item[C)] No Change
\end{itemize}
$\Rightarrow$ C

\subsection{Multiple Choice \#2}
A parallel plate capacitor with initial capacitance C is hooked up to a battery of voltage V.  If I increase the plate separation, how will the capacitance (C) change?

\begin{itemize}
\item[A)] Increase
\item[B)] Decrease
\item[C)] No Change
\end{itemize}
$\Rightarrow$ B

\subsection{Multiple Choice \#3}
A parallel plate capacitor with initial capacitance C is hooked up to a battery of voltage V.  If I increase the plate separation, how will the energy stored in the capacitor change?

\begin{itemize}
\item[A)] Increase
\item[B)] Decrease
\item[C)] No Change
\end{itemize}
$\Rightarrow$ B

\subsection{Multiple Choice \#4}
A parallel plate capacitor with initial capacitance C is hooked up to a battery of voltage V.  If I increase the plate area, how will the charge stored in the capacitor change?

\begin{itemize}
\item[A)] Increase
\item[B)] Decrease
\item[C)] No Change
\end{itemize}
$\Rightarrow$ A

\pagebreak
\subsection{Multiple Choice \#5}
A parallel plate capacitor with initial capacitance C is hooked up to a battery of voltage V.  If I decrease the voltage (V), how will the charge stored in the capacitor change?

\begin{itemize}
\item[A)] Increase
\item[B)] Decrease
\item[C)] No Change
\end{itemize}
$\Rightarrow$ B

\subsection{Multiple Choice \#6}
A parallel plate capacitor with initial capacitance C is hooked up to a battery of voltage V.  If I decrease the plate area, how will the energy stored in the capacitor change?

\begin{itemize}
\item[A)] Increase
\item[B)] Decrease
\item[C)] No Change
\end{itemize}
$\Rightarrow$ B

\subsection{Change in Potential and Energy}
Suppose I connect a battery of voltage V and a capacitor of capacitance C in series.  Initially, this capacitor has no dielectric material between the plates.  Now, suppose I insert a dielectric material with constant $\kappa$ in-between the plates.  What is the new voltage across the capacitor?  What is the new energy stored in the capacitor? \\* \\*
$\Rightarrow V = V_{Original}$, \hspace{2mm}$E = \kappa E_{Original}$ 

%%%%%%%%%%%%%%%%%%%%%%%%%%%%%%%%%%%%%%%%%%%%%%%%%%%%%%%%%%%%%%%%%%%%%%%%%%%%%%%%%%%%%%%%%%%%

\end{document}



