\title{Introductory Electricity, Magnetism, and Optics Practice Problems}
\author{Cyrus Vandrevala}

\documentclass[11pt]{article}
\usepackage[margin=2.5cm]{geometry}
\usepackage[pdftex]{graphicx}

\begin{document}

\maketitle
\tableofcontents
\vspace{10pt}

\subsection*{Useful Constants}
Electron Mass = $9.11 \times 10^{-31}$ kg \\*
Proton Mass = $1.67 \times 10^{-27}$ kg \\*
Elementary Charge = $1.602 \times 10^{-19}$ C \\*
Coulomb's Constant = $8.99 \times 10^9$ Nm$^2$/C$^2$ \\*
Avogadro's Number = $ 6.02 \times 10^{23}$ atoms/mole \\*
Radius of Gold Nucleus = 6.6 fm

%%%%%%%%%%%%%%%%%%%%%%%%%%%%%%%%%%%%%%%%%%%%%%%%%%%%%%%%%%%%%%%%%%%%%%%%%%%%%%%%%%%%%%%%%%%%

\pagebreak
\section{Gauss's Law}

\subsection{True or False?}
True or False?  Gauss's Law is only valid when we are calculating the electric field around spherical objects.

\subsection{Offset Hole}
I have a solid conducting sphere of radius R.  I make a small hole inside the sphere and place a charge of -Q inside.  The hole is NOT at the center of the sphere – it is located at a distance of R/2 from the center.  What is the net electric flux passing through the surface of the sphere?

\subsection{Irregularly Shaped Conductor}
Suppose I have a charged conductor with some irregular shape.  What is the orientation of the electric field close to the surface of the conductor?  What is its magnitude

\subsection{Concentric Spheres}
Suppose I place an insulating ball with radius $R_1$ inside of a thick, conducting shell with inner radius $R_{Inner}$ and outer radius $R_{Outer}$.  There is nothing between the objects.  I am told that the electric field is $E = +kQ/r^2$ at a radius greater than r2 (i.e. outside the outer shell).  What can you tell me about the charge on the inner insulating ball?

\subsection{Multiple Choice \#1}
Which of the following statements is NOT true?

\begin{itemize}
 \item[A)] Gauss's Law works for any charge distribution, but it is only convenient to use when there is a high degree of symmetry in the problem
 \item[B)] A Gaussian surface must be a closed surface
 \item[C)] Although it is normally used for point charges, Coulomb's law can be modified to determine the electric field due to a continuous charge distribution
 \item[D)] When I use Gauss's Law, if the charge within my Gaussian surface is zero, then the electric field at every point on the surface must be zero
\end{itemize}

\subsection{Multiple Choice \#2}
Suppose I enclose a perfect dipole in a spherical Gaussian surface.  Which of the following statements is false?

\begin{itemize}
\item[A)] The net electric flux through the surface is zero
\item[B)] The electric field is zero everywhere on the surface
\item[C)] The electric flux is not zero at every point on the surface
\item[D)] The net enclosed charge equals zero
\item[E)] The net electric flux would stay the same if I switched the positive and negative charges, keeping everything else constant
\end{itemize}

%%%%%%%%%%%%%%%%%%%%%%%%%%%%%%%%%%%%%%%%%%%%%%%%%%%%%%%%%%%%%%%%%%%%%%%%%%%%%%%%%%%%%%%%%%%%

\pagebreak
\section{Electric Potential}

\subsection{Charging Spheres \#1}
Suppose I have two identical conducting spheres, each of radius R, each charged to +1 C.  I touch a third, uncharged conducting sphere of radius R' to one of the original two, and then the other.  What is the final charge on the third sphere?

\subsection{Charging Spheres \#2}
Suppose I have conducting sphere A with radius $R_A=$ 10 cm and conducting sphere B with radius $R_B=$ 20 cm.  I charge sphere A up to 60 $\mu$C and leave sphere B uncharged.  I then connect the two spheres with a thin wire and let them come to equilibrium.  After I disconnect the wire, what is the final charge on sphere B?  You may assume that none of the charge got “stuck” on the thin wire.

\subsection{Charges on a Square}
Four point charges, each with charge -Q, are placed at the corners of a square with side length d.  What is the magnitude of the net electric potential at the center of the square?  Assume V = 0 V at infinite distance.

\subsection{Electric Field to Electric Potential}
The electric field in some region of space is equal to $\vec{E} = 2 \hat{x} N/C$.  Calculate the potential difference between the point (2,0,0) and the point (1,1,1).

\subsection{Electric Potential to Electric Field}
Suppose I have an electric potential given by V(x,y,z) = 5x + 3xy – 8z V.  Recall that (x,y,z) are variables, not unit vectors.  What is the magnitude and direction of the electric field at the point (1,1,0)?

\subsection{Electric Potential to Electric Force}
A region of space has an electric potential given by V = xy + yz + xyz.  I place an electron at the point (2,1,1).  What is the magnitude of the electric force that it feels at that point?

%%%%%%%%%%%%%%%%%%%%%%%%%%%%%%%%%%%%%%%%%%%%%%%%%%%%%%%%%%%%%%%%%%%%%%%%%%%%%%%%%%%%%%%%%%%%

\pagebreak
\section{Electric Potential Energy}

\subsection{Charges on an Equilateral Triangle}
I place three point charges, +Q, +2Q, and -Q, at the corners of an equilateral triangle with side length L = 1 m.  What is the total electric potential energy of the system?

\subsection{Charges on a Square}
I place four charges - +1 $\mu$C, -2 $\mu$C, +3 $\mu$C, and +5 $\mu$C - at the corners of a square with side length D = 10 cm.  How much work would it take to move a fifth charge (q = -7 $\mu$C) from infinite distance to the center of the square?  Assume that V = 0 V at infinite distance.

\subsection{Particle Collider}
Suppose you are a scientist working at the Large Hadron Collider (LHC) at CERN.  You will be shooting two gold nuclei together at high speed so that they will collide and fracture.  In this experiment, you will hold one gold nucleus still and smash the other one into it.  Estimate how much energy you need to fire the second gold nucleus at so that the two nuclei just barely make contact.

%%%%%%%%%%%%%%%%%%%%%%%%%%%%%%%%%%%%%%%%%%%%%%%%%%%%%%%%%%%%%%%%%%%%%%%%%%%%%%%%%%%%%%%%%%%%

\pagebreak
\section{Capacitors}

%%%%%%%%%%%%%%%%%%%%%%%%%%%%%%%%%%%%%%%%%%%%%%%%%%%%%%%%%%%%%%%%%%%%%%%%%%%%%%%%%%%%%%%%%%%%

\pagebreak
\section{Resistors}

%%%%%%%%%%%%%%%%%%%%%%%%%%%%%%%%%%%%%%%%%%%%%%%%%%%%%%%%%%%%%%%%%%%%%%%%%%%%%%%%%%%%%%%%%%%%

\pagebreak
\section{Kirichoff's Rules}

%%%%%%%%%%%%%%%%%%%%%%%%%%%%%%%%%%%%%%%%%%%%%%%%%%%%%%%%%%%%%%%%%%%%%%%%%%%%%%%%%%%%%%%%%%%%


\pagebreak
\section{Answers}
\hspace{1cm}

\paragraph{1.1 $\rightarrow$} Coulomb
\paragraph{1.2 $\rightarrow$} $-1.602 \times 10^{-19}$ C
\paragraph{1.3 $\rightarrow$} $2.749 \times 10^{26}$ electrons

\paragraph{2.1 $\rightarrow$} $8.214 \times 10^{-8}$ N
\paragraph{2.2 $\rightarrow$} 0 N
\paragraph{2.3 $\rightarrow$} 
\paragraph{2.4 $\rightarrow$}
\paragraph{2.5 $\rightarrow$}
\paragraph{2.6 $\rightarrow$}

\paragraph{3.1 $\rightarrow$} Induction
\paragraph{3.2 $\rightarrow$} Conduction/Contact
\paragraph{3.3 $\rightarrow$} 
\paragraph{3.4 $\rightarrow$} 
\paragraph{3.5 $\rightarrow$} 
\paragraph{3.6 $\rightarrow$} 

\paragraph{4.1 $\rightarrow$} $-\hat{x}$ direction
\paragraph{4.2 $\rightarrow$} $\vec{E} = \frac{kQz}{(z^2 + R^2)^{3/2}}\hat{z}$
\paragraph{4.3 $\rightarrow$} 


\end{document}