\title{Introductory E\&M Practice Problems}
\author{Cyrus Vandrevala}

\documentclass[12pt]{article}

\begin{document}
\maketitle

\section*{Topics Covered}
\begin{itemize}
\item Electric Flux
\item Electric Potential
\item Electric Potential Energy\\*
\end{itemize}

\section*{Question \#1}
Consider a sphere of radius R.  I place the sphere into a uniform electric field E.  What is the electric flux passing through the sphere?

\section*{Question \#2}
Which of the following statements is NOT true?

\begin{itemize}
 Gauss's Law works for any charge distribution, but it is only convenient to use when there is a high degree of symmetry in the problem
 A Gaussian surface must be a closed surface
 Although it is normally used for point charges, Coulomb's law can be modified to determine the electric field due to a continuous charge distribution
 When I use Gauss's Law, if the charge within my Gaussian surface is zero, then the electric field at every point on the surface must be zero
\end{itemize}

\section{Question \#3}
Charge density is defined as the _____ per unit length/area/volume.

\begin{itemize}
\item Electric Field
\item Charge
\item Electric Force
\item Electric Dipole Moment
\item None of the Above
\end{itemize}

\section*{Question \#4}
Suppose I have two identical conducting spheres, each charged to +1 C.  I touch a third, identical, uncharged conducting sphere to one, then the other.  What is the final charge on the third sphere?

\section*{Question \#5}
Suppose I enclose a perfect dipole in a spherical Gaussian surface.  Which of the following statements is false?

\begin{itemize}
\item The net electric flux through the surface is zero
\item The electric field is zero everywhere on the surface
\item The electric flux is not zero at every point on the surface
\item The net enclosed charge equals zero
\item  The net electric flux would stay the same if I switched the positive and negative charges, keeping everything else constant
\end{itemize}

\section*{Question \#6}
Some region of space contains a uniform electric field given by $\vec{E} = 3\hat{i} + 4\hat{j}$ N/C.  I place a metal spherical shell of radius R = 1 m into the electric field.  What is the net electric flux through the sphere?

\section*{Question \#7}
True or False?  Gauss's Law is only valid when we are calculating the electric field around spherical objects.

\section*{Question \#8}
Suppose I have a charged conductor with some irregular shape.  What is the orientation of the electric field close to the surface of the conductor?

\section*{Question \#9}
Suppose I place an insulating ball with radius $R_1$ inside of a thick, conducting shell with inner radius $R_{Inner}$ and outer radius $R_{Outer}$.  There is nothing between the objects.  I am told that the electric field is E = +kQ/r^2 at a radius greater than r2 (i.e. outside the outer shell).  What can you tell me about the charge on the inner insulating ball?

\section*{Question \#10}
I place three point charges, +Q, +2Q, and -Q, at the corners of an equilateral triangle with side length $L = 1 m$.  What is the total electric potential energy of the system?

\section*{Question \#11}
I have a solid conducting sphere of radius R.  I make a small hole inside the sphere and place a charge of -Q inside.  The hole is NOT centered – it is located at R/2.  
What is the electric flux passing through the surface of the sphere?

\section*{Question \#12}
The electric field in some region of space is equal to:
E = 2x N/C (x is a unit vector)
Calculate the change in voltage between the origin and the point (1,1,1)

\section*{Question \#13}
A region of space has an electric potential:
V = x + y + z (x,y,z are variables, NOT unit vectors)
I place an electron at the point (2,1,1).  What is the magnitude of the electric force that it feels at that point?  Assume Q is the magnitude of the charge on an electron (i.e. Q = +1.6E-19).



\end{document}