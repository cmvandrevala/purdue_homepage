\title{Introductory E\&M Practice Problems}
\author{Cyrus Vandrevala}

\documentclass[12pt]{article}
\usepackage[margin=0.9in]{geometry}

\begin{document}

\maketitle
\tableofcontents
\hspace{30mm}

\section*{Useful Constants}
Electron Mass = $9.11 \times 10^{-31}$ kg \\*
Proton Mass = $1.67 \times 10^{-27}$ kg \\*
Elementary Charge = $1.602 \times 10^{-19}$ C \\*
Coulomb's Constant = $8.99 \times 10^9$ Nm$^2$/C$^2$ \\*
Avogadro's Number = $ 6.02 \times 10^{23}$ atoms/mole \\*
Atomic Mass of Copper = 29 \\*
Molar Mass of Copper = 55.8 g/mole \\*

\pagebreak

\section{Charges in Magnetic Fields}

\subsection{Right Hand Rule \#1}
Suppose a magnetic field is pointing in the +x-direction.  I fire a positively charged particle in the -y-direction.  The force on the particle acts in which direction?

\subsection{Right Hand Rule \#2}
Suppose a magnetic field is pointing in the -y-direction.  I fire a negatively charged particle in the -z-direction.  The force on the particle acts in which direction?

\subsection{Right Hand Rule \#3}
Suppose a magnetic field is pointing in the +x-direction.  I fire a positively charged particle in the -x-direction.  The force on the particle acts in which direction?

\subsection{Right Hand Rule \#4}
Suppose a magnetic field is pointing in the +z-direction.  I fire a positively charged particle in the +z-direction.  The force on the particle acts in which direction?

\subsection*{Particle Paths}
A charged particle in a uniform magnetic field can NOT experience a path in the shape of a:

\begin{itemize}
	\item Circle
	\item Straight Line
	\item Helix
	\item Ellipse
\end{itemize}


\subsection{Magnetic Force}
A charged particle with charge +0.5 C is fired perpendicularly to a 0.01 T magnetic field at 100 m/s.  What is the magnitude of the force on the particle?


\section{Coulomb's Force Law}

\subsection{Hydrogen Atom}
In a hydrogen atom, a proton is separated from an electron by an average distance of about $5.3 \times 10^{-11}$ meters.  Calculate the electrostatic force of attraction by the proton on the electron.  Do the same for the electron on the proton.

\subsection{Force at the Center of a Square}
Suppose I place four charges (each +Q) at the four vertices of a square with side length L = 10 cm.  What is the magnitude of the net force on a positive point charge (+q) located at the center of the square?


\section{Charging Objects}
\section{Electric Field}
\section{Electric Dipoles}



\section*{Question \#2}
Suppose I place a charge of Q1 = +1 C at the point (1 m, 0 m) and a charge of Q2 = -2 C at the point (0 m, 0 m).  At what point in the xy-plane could I put a negative charge of Q3 = -5 C such that Q3 would feel no net electrostatic force?\\*\\*


\section*{Charging a Conducting Sphere}
I bring a charged insulator close to an uncharged conductor (not touching).  I then ground the conductor.  This method of charging the conductor is called charging by \underline{\hspace{1cm}}.



\section*{Question \#6}
An electron is fired into a uniform electric field.  The initial velocity of the electron is given by:
$\vec{v} = 500 \hat{x} + 100 \hat{y} - 300 \hat{z}$

 given by:
$\vec{E} = 100 \hat{x} + 200 \hat{y} - 150 \hat{z}$

\section*{Question \#7}
Consider an electric dipole with charges +Q and -Q located at x = +1 m and x = -1 m respectively.  Calculate the net electric field at x = 0 m.

\section*{Question \#8}
Which of the following statements is NOT true?

\begin{itemize}
\item If a charged particle with non-zero charge is put into a non-zero electric field, the particle must experience a force
 A perfect electric dipole consists of two point charges separated by a small distance
 Electric field lines emanate from positive charges and terminate on negative charges
 Electric field lines that are closely spaced indicate a strong field, while those that are far apart indicate a weak field
\end{itemize}

\section*{Question \#9}
A place a dipole with dipole moment p = 10 Cm [+z] in an electric field E = 100 N/C [-x].  What is the potential energy of the dipole?

\section*{Question \#10}
Consider a charged ring in the xy-plane, centered at the origin, with charge density D = cos(theta) where theta is the angle about the ring in standard orientation.  If I place a positive charge at the center of the ring, which way will it move?

\section*{Question \#11}
An electron is fired into a region of uniform electric field given by $\vec{E} = 100\hat{y}$. At time t = 0 s, it is located at the origin with an initial velocity of $\vec{v_o} = 5.0 \times 10^5 \hat{x} + 3 \times 10^5 \hat{y} m/s$.  What is the velocity of the particle at time t = 10 s?

\section*{Question \#12}
An electron is fired into a region of uniform electric field given by $\vec{E} = 100\hat{y}$. At time t = 0 s, it is located at the origin with an initial velocity of $\vec{v_o} = 5.0 \times 10^5 \hat{x} + 3 \times 10^5 \hat{y} m/s$.  At what point does it intersect the x-axis?

\section*{Question \#13}
Suppose I put a polar object with dipole moment 2 Cm (not centimeters) in a uniform electric field of magnitude 300 N/C.  What is the maximum possible potential energy of the system?

\section*{Question \#14}
Fusion reactors require you to smash together two positively charged atomic nuclei in order for them to fuse and release energy.  Why is this so difficult?

\section*{Question \#15}
Why is it important, while soldering components on a circuit board, not to leave any pointy protrusions on the solder?

\section*{Answers}

\begin{itemize}
\item Question \#1: ..... N
\item Question \#2: ..... N
\item $8 \times 10^{-5}$ N
\item $8 \times 10^{-5}$ N
\item $8 \times 10^{-5}$ N
\item $3$
\item $8 \times 10^{-5}$ N
\item $8 \times 10^{-5}$ N
\item $8 \times 10^{-5}$ N
\item $3$
\item $8 \times 10^{-5}$ N
\item $8 \times 10^{-5}$ N
\item $8 \times 10^{-5}$ N
\item $2$
\end{itemize}







%
%\begin{itemize}
%\item $8 \times 10^{-5}$ N
%\item $8 \times 10^{-5}$ N
%\item $8 \times 10^{-5}$ N
%\item $8 \times 10^{-5}$ N
%\item $8 \times 10^{-5}$ N
%\end{itemize}

\end{document}