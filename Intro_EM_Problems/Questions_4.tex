\title{Introductory E\&M Practice Problems}
\author{Cyrus Vandrevala}

\documentclass[12pt]{article}
\usepackage[margin=0.9in]{geometry}

\begin{document}

\maketitle
\tableofcontents
\hspace{30mm}

\section*{Useful Constants}
Electron Mass = $9.11 \times 10^{-31}$ kg \\*
Proton Mass = $1.67 \times 10^{-27}$ kg \\*
Elementary Charge = $1.602 \times 10^{-19}$ C \\*
Coulomb's Constant = $8.99 \times 10^9$ Nm$^2$/C$^2$ \\*
Avogadro's Number = $ 6.02 \times 10^{23}$ atoms/mole \\*
Atomic Mass of Copper = 29 \\*
Molar Mass of Copper = 55.8 g/mole \\*

%%%%%%%%%%%%%%%%%%%%%%%%%%%%%%%%%%%%%%%%%%%%%%%%%%%%%%%%%%%%%%%%%%%%%%%%%%%%%%%%%%%%%%%%%%%%

\pagebreak
\section{Charges in Magnetic Fields}

\subsection{Right Hand Rule \#1}
Suppose a magnetic field is pointing in the +x-direction.  I fire a positively charged particle in the -y-direction.  The force on the particle acts in which direction?

\subsection{Right Hand Rule \#2}
Suppose a magnetic field is pointing in the -y-direction.  I fire a negatively charged particle in the -z-direction.  The force on the particle acts in which direction?

\subsection{Right Hand Rule \#3}
Suppose a magnetic field is pointing in the +x-direction.  I fire a positively charged particle in the -x-direction.  The force on the particle acts in which direction?

\subsection{Right Hand Rule \#4}
Suppose a magnetic field is pointing in the +z-direction.  I fire a positively charged particle in the +z-direction.  The force on the particle acts in which direction?

\subsection{Magnetic Force}
A magnetic field is given by $\vec{B} = 5 \hat{x} - 8\hat{y} + 3\hat{z}$ T.  A proton is launched into the field with an initial velocity of $\vec{v_o} = 2\hat{x} - 8\hat{y} - 9\hat{z}$ m/s.  What is the magnetic force on the particle?

\subsection{Circular Motion}
A magnetic field is given by $\vec{B} = 5 \hat{x} - 3\hat{y}$ T.  A proton is launched into the field with an initial velocity of $\vec{v_o} = 8\hat{y} - 9\hat{z}$ m/s.  What is the radius of the circle traced out by the particle?  What is the period of motion?

\subsection{Particle Paths}
A charged particle in a uniform magnetic field can NOT experience a path in the shape of a:

\begin{itemize}
	\item Circle
	\item Straight Line
	\item Helix
	\item Ellipse
\end{itemize}

%%%%%%%%%%%%%%%%%%%%%%%%%%%%%%%%%%%%%%%%%%%%%%%%%%%%%%%%%%%%%%%%%%%%%%%%%%%%%%%%%%%%%%%%%%%%

\pagebreak
\section{Sources of Magnetic Fields}

\subsection{Magnetic Force}
A charged particle with charge +0.5 C is fired perpendicularly to a 0.01 T magnetic field at 100 m/s.  What is the magnitude of the force on the particle?

%%%%%%%%%%%%%%%%%%%%%%%%%%%%%%%%%%%%%%%%%%%%%%%%%%%%%%%%%%%%%%%%%%%%%%%%%%%%%%%%%%%%%%%%%%%%

\pagebreak
\section{Answers}

\paragraph{1.1 $\rightarrow$} Question \#1: ..... N


\end{document}