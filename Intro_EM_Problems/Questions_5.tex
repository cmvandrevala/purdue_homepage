\title{Introductory E\&M Practice Problems}
\author{Cyrus Vandrevala}

\documentclass[11pt]{article}
\usepackage[margin=1.0in]{geometry}
\usepackage{setspace}

\begin{document}

\maketitle
\tableofcontents
\hspace{30mm}

\subsection*{Useful Constants}
Electron Mass = $9.11 \times 10^{-31}$ kg \\*
Proton Mass = $1.67 \times 10^{-27}$ kg \\*
Elementary Charge = $1.602 \times 10^{-19}$ C \\*
Coulomb's Constant = $8.99 \times 10^9$ Nm$^2$/C$^2$ \\*
Avogadro's Number = $ 6.02 \times 10^{23}$ atoms/mole \\*
Atomic Number of Copper = 29 \\*
Molar Mass of Copper = 63.5 g/mole \\*

%%%%%%%%%%%%%%%%%%%%%%%%%%%%%%%%%%%%%%%%%%%%%%%%%%%%%%%%%%%%%%%%%%%%%%%%%%%%%%%%%%%%%%%%%%%%

\pagebreak
\section{Quantized Electric Charge}

\subsection{Units of Charge}
What is the fundamental unit of electric charge?

\subsection{Elementary Charge}
What is the charge of one electron?

\subsection{Number of Electrons in Copper}
Estimate the number of electrons in a 1.0 kg block of copper that has been charged to +10 $\mu$C.

%%%%%%%%%%%%%%%%%%%%%%%%%%%%%%%%%%%%%%%%%%%%%%%%%%%%%%%%%%%%%%%%%%%%%%%%%%%%%%%%%%%%%%%%%%%%

\pagebreak
\section{Coulomb's Force Law}

\subsection{Hydrogen Atom}
In a hydrogen atom, a proton is separated from an electron by an average distance of about $5.3 \times 10^{-11}$ meters.  Calculate the electrostatic force of attraction by the proton on the electron.  Do the same for the electron on the proton.

\subsection{Force at the Center of a Square}
Suppose I place four charges (each +Q) at the four vertices of a square with side length L = 10 cm.  What is the magnitude of the net force on a positive point charge (+q) located at the center of the square?

\subsection{Equilibrium Point}
Suppose I place a charge of Q1 = +1 C at the point (1 m, 0 m) and a charge of Q2 = -2 C at the point (0 m, 0 m).  At what point in the xy-plane could I put a negative charge of Q3 = -5 C such that Q3 would feel no net electrostatic force?

\subsection{Particle Colliders}
Fusion reactors require you to smash together two positively charged atomic nuclei in order for them to fuse and release energy.  Why is this so difficult?

%%%%%%%%%%%%%%%%%%%%%%%%%%%%%%%%%%%%%%%%%%%%%%%%%%%%%%%%%%%%%%%%%%%%%%%%%%%%%%%%%%%%%%%%%%%%

\pagebreak
\section{Charging Objects}

\subsection{Charging a Conducting Sphere \#1}
I bring a charged insulator close to an uncharged conductor (not touching).  I then ground the conductor.  This method of charging the conductor is called charging by \underline{\hspace{1cm}}.

\subsection{Charging a Conducting Sphere \#2}
I touch an uncharged conductor to a second negatively charged conductor.  This method of charging an object is called charging by \underline{\hspace{1cm}}.

\subsection{Charging a Glass Rod}
I rub a glass rod against a silk cloth in order to charge up the glass rod.  This is an example of charging by \underline{\hspace{1cm}}.


%%%%%%%%%%%%%%%%%%%%%%%%%%%%%%%%%%%%%%%%%%%%%%%%%%%%%%%%%%%%%%%%%%%%%%%%%%%%%%%%%%%%%%%%%%%%

\pagebreak
\section{Continuous Electric Charge}

\subsection{Charge at the Center of a Ring}
Consider a charged ring in the xy-plane, centered at the origin, with charge density D = cos$\theta$ where $\theta$ is the angle about the ring in standard orientation.  If I place a positive charge at the center of the ring, which way will it move?

\subsection{Uniformly Charged Ring}
Calculate the electric field along the axis of a uniformly charged ring (radius R, net charge Q).

\subsection{Uniformly Charged Ring (Difficult Extra Credit)}
Show that for small distances, the electric field around the center of the uniformly charged ring is linear with distance.  Then, show that a particle released near the center of the ring experiences simple harmonic motion.

\subsection{Charged Line \#1}
A uniform line of charge extends from the point (0,0) to (1,0).  It has a net charge of +Q.  What is the electric field at the point (2,0)?

\subsection{Charged Line \#2}
A uniform line of charge extends from the point (0,0) to (1,0).  It has a net charge of +Q.  What is the electric field at the point (2,1)?

%%%%%%%%%%%%%%%%%%%%%%%%%%%%%%%%%%%%%%%%%%%%%%%%%%%%%%%%%%%%%%%%%%%%%%%%%%%%%%%%%%%%%%%%%%%%

\pagebreak
\section{Electric Field}

\subsection{Electron in an Electric Field \#1}
An electron is fired into a uniform electric field.  The initial velocity of the electron is given by $\vec{v} = 500 \hat{x} + 100 \hat{y} - 300 \hat{z}$, and the electric field is given by $\vec{E} = 100 \hat{x} + 200 \hat{y} - 150 \hat{z}$.  Calculate the magnitude of the acceleration of the electron.

\subsection{Electron in an Electric Field \#2}
An electron is fired into a region of uniform electric field given by $\vec{E} = 100\hat{y}$. At time t = 0 s, it is located at the origin with an initial velocity of $\vec{v_o} = 5.0 \times 10^5 \hat{x} + 3 \times 10^5 \hat{y}$ m/s.  What is the velocity of the particle at time t = 10 s?

\subsection{Electron in an Electric Field \#3}
An electron is fired into a region of uniform electric field given by $\vec{E} = 100\hat{y}$. At time t = 0 s, it is located at the origin with an initial velocity of $\vec{v_o} = 5.0 \times 10^5 \hat{x} + 3 \times 10^5 \hat{y}$ m/s.  At what point does it intersect the x-axis?

\subsection{Multiple Choice}
Which of the following statements is NOT true?

\begin{itemize}
	\item[A)] Electric field lines emanate from positive charges and terminate on negative charges.
	\item[B)] Electric field lines that are closely spaced indicate a strong field, while those that are far apart indicate a weak field.
	\item[C)] Positive point charges feel an electric force parallel to electric field lines.
	\item[D)] A positive point charge does NOT produce an electric field.
\end{itemize}

%%%%%%%%%%%%%%%%%%%%%%%%%%%%%%%%%%%%%%%%%%%%%%%%%%%%%%%%%%%%%%%%%%%%%%%%%%%%%%%%%%%%%%%%%%%%

\pagebreak
\section{Electric Dipoles}

\subsection{Electric Field Due to a Dipole}
Consider an electric dipole with charges +Q and -Q located at x = +1 m and x = -1 m respectively.  Calculate the net electric field at x = 0 m.

\subsection{Potential Energy of an Electric Dipole}
A place a dipole with dipole moment p = 10 Cm [+z] in an electric field E = 100 N/C [-x].  What is the potential energy of the dipole?

\subsection{Maximum Potential Energy of an Electric Dipole}
Suppose I put a polar object with dipole moment 2 Cm in a uniform electric field of magnitude 300 N/C.  What is the maximum possible value for the potential energy of the system?

\subsection{Multiple Choice}
Which of the following statements is NOT true?

\begin{itemize}
	\item[A)] If a perfect electric dipole is placed into a uniform electric field, the dipole can experience a net force.
	\item[B)] A perfect electric dipole consists of two equal and opposite point charges separated by a small distance.
	\item[C)] If a perfect electric dipole is placed into a non-uniform electric field, the dipole can experience a net force.
	\item[D)] If a perfect electric dipole is placed into a uniform electric field, the dipole can experience a net torque.
	\item[E)] If a perfect electric dipole is placed into a non-uniform electric field, the dipole can experience a net force.
\end{itemize}

%%%%%%%%%%%%%%%%%%%%%%%%%%%%%%%%%%%%%%%%%%%%%%%%%%%%%%%%%%%%%%%%%%%%%%%%%%%%%%%%%%%%%%%%%%%%

\pagebreak
\section{Answers}
\hspace{1cm}
\begin{spacing}{0.5}

\paragraph{1.1 $\rightarrow$} Coulomb
\paragraph{1.2 $\rightarrow$} $-1.602 \times 10^{-19}$ C
\paragraph{1.3 $\rightarrow$} $2.749 \times 10^{26}$ electrons

\paragraph{2.1 $\rightarrow$} $8.214 \times 10^{-8}$ N
\paragraph{2.2 $\rightarrow$} 0 N
\paragraph{2.3 $\rightarrow$} 
\paragraph{2.4 $\rightarrow$} In order to get two nuclei to fuse, one must launch them toward each other with a speed that can overcome the Coulomb repulsive force.

\paragraph{3.1 $\rightarrow$} Induction
\paragraph{3.2 $\rightarrow$} Conduction/Contact
\paragraph{3.3 $\rightarrow$} Friction

\paragraph{4.1 $\rightarrow$} $-\hat{x}$ direction
\paragraph{4.2 $\rightarrow$} $\vec{E} = \frac{kQz}{(z^2 + R^2)^{3/2}}\hat{z}$
\paragraph{4.3 $\rightarrow$} $\vec{E} \approx \frac{kQz}{R^3} \rightarrow \vec{F} = m \vec{a} \Rightarrow -\frac{kQz}{R^3} = m\frac{d^2z}{dt^2} \Rightarrow \frac{d^2z}{dt^2} + \frac{kQz}{mR^3} = 0 \Rightarrow z(t) = Asin(\omega t)$ where $\omega = \sqrt{\frac{kQ}{mR^3}}$
\paragraph{4.4 $\rightarrow$}
\paragraph{4.5 $\rightarrow$}

\paragraph{5.1 $\rightarrow$} $\vec{a} = 4.735 \times 10^{13}$ m/s$^2$
\paragraph{5.2 $\rightarrow$}
\paragraph{5.3 $\rightarrow$}
\paragraph{5.4 $\rightarrow$} D

\paragraph{6.1 $\rightarrow$} $-2kQ \hat{x}$
\paragraph{6.2 $\rightarrow$} 0 J
\paragraph{6.3 $\rightarrow$} 600 Nm
\paragraph{6.4 $\rightarrow$} A

\end{spacing}
\end{document}