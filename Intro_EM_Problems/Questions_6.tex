\title{Introductory E\&M Practice Problems}
\author{Cyrus Vandrevala}

\documentclass[11pt]{article}
\usepackage[margin=1.0in]{geometry}
\usepackage{setspace}

\begin{document}

\maketitle
\tableofcontents
\hspace{30mm}

\subsection*{Useful Constants}
Electron Mass = $9.11 \times 10^{-31}$ kg \\*
Proton Mass = $1.67 \times 10^{-27}$ kg \\*
Elementary Charge = $1.602 \times 10^{-19}$ C \\*
Coulomb's Constant = $8.99 \times 10^9$ Nm$^2$/C$^2$ \\*

%%%%%%%%%%%%%%%%%%%%%%%%%%%%%%%%%%%%%%%%%%%%%%%%%%%%%%%%%%%%%%%%%%%%%%%%%%%%%%%%%%%%%%%%%%%%

\pagebreak
\section{Mirrors}

\subsection{Mirror Problem \#1}
A converging mirror has a focal length of |f| = 10 cm.  I place an object with a height of 6 cm, 30 cm in front of the mirror.  

\begin{itemize}
\item What is the final image distance?
\item What is the final image height?
\item Is it in front of the mirror or behind the mirror?  
\item Is it real or virtual?
\item Is it upright or inverted?
\end{itemize}

\subsection{Mirror Problem \#2}
A diverging mirror has a focal length of |f| = 15 cm.  I place an object 10 cm in front of the mirror.  

\begin{itemize}
\item What is the final image distance?
\item What is the final image height?
\item Is it in front of the mirror or behind the mirror?  
\item Is it real or virtual?
\item Is it upright or inverted?
\end{itemize}

\subsection{Mirror Problem \#3}
A converging mirror has a focal length of |f| = 40 cm.  I place an object with height 5 cm, 10 cm in front of the mirror.  

\begin{itemize}
\item What is the final image distance?
\item What is the final image height?
\item Is it in front of the mirror or behind the mirror?  
\item Is it real or virtual?
\item Is it upright or inverted?
\end{itemize}

\subsection{Mirror Problem \#4}
A diverging mirror has a focal length of |f| = 20 cm.  I place an object with height 12 cm, 30 cm in front of the mirror.  

\begin{itemize}
\item What is the final image distance?
\item What is the final image height?
\item Is it in front of the mirror or behind the mirror?  
\item Is it real or virtual?
\item Is it upright or inverted?
\end{itemize}
%%%%%%%%%%%%%%%%%%%%%%%%%%%%%%%%%%%%%%%%%%%%%%%%%%%%%%%%%%%%%%%%%%%%%%%%%%%%%%%%%%%%%%%%%%%%

\pagebreak
\section{Lenses}

\subsection{Lens Problem \#1}
A converging mirror has a focal length of |f| = 10 cm.  I place an object with a height of 6 cm, 30 cm in front of the mirror.  

\begin{itemize}
\item What is the final image distance?
\item What is the final image height?
\item Is it in front of the mirror or behind the mirror?  
\item Is it real or virtual?
\item Is it upright or inverted?
\end{itemize}

\subsection{Lens Problem \#2}
A diverging mirror has a focal length of |f| = 15 cm.  I place an object 10 cm in front of the mirror.  

\begin{itemize}
\item What is the final image distance?
\item What is the final image height?
\item Is it in front of the mirror or behind the mirror?  
\item Is it real or virtual?
\item Is it upright or inverted?
\end{itemize}

\subsection{Lens Problem \#3}
A converging mirror has a focal length of |f| = 40 cm.  I place an object with height 5 cm, 10 cm in front of the mirror.  

\begin{itemize}
\item What is the final image distance?
\item What is the final image height?
\item Is it in front of the mirror or behind the mirror?  
\item Is it real or virtual?
\item Is it upright or inverted?
\end{itemize}

\subsection{Lens Problem \#4}
A diverging mirror has a focal length of |f| = 20 cm.  I place an object with height 12 cm, 30 cm in front of the mirror.  

\begin{itemize}
\item What is the final image distance?
\item What is the final image height?
\item Is it in front of the mirror or behind the mirror?  
\item Is it real or virtual?
\item Is it upright or inverted?
\end{itemize}



%%%%%%%%%%%%%%%%%%%%%%%%%%%%%%%%%%%%%%%%%%%%%%%%%%%%%%%%%%%%%%%%%%%%%%%%%%%%%%%%%%%%%%%%%%%%


\pagebreak
\section{Interference Effects}

\subsection{Double Slit Experiment}
Consider an electric dipole with charges +Q and -Q located at x = +1 m and x = -1 m respectively.  Calculate the net electric field at x = 0 m.

\subsection{Potential Energy of an Electric Dipole}
A place a dipole with dipole moment p = 10 Cm [+z] in an electric field E = 100 N/C [-x].  What is the potential energy of the dipole?

\subsection{Maximum Potential Energy of an Electric Dipole}
Suppose I put a polar object with dipole moment 2 Cm in a uniform electric field of magnitude 300 N/C.  What is the maximum possible value for the potential energy of the system?

\subsection{Multiple Choice}
Which of the following statements is NOT true?

\begin{itemize}
	\item[A)] If a perfect electric dipole is placed into a uniform electric field, the dipole can experience a net force.
	\item[B)] A perfect electric dipole consists of two equal and opposite point charges separated by a small distance.
	\item[C)] If a perfect electric dipole is placed into a non-uniform electric field, the dipole can experience a net force.
	\item[D)] If a perfect electric dipole is placed into a uniform electric field, the dipole can experience a net torque.
	\item[E)] If a perfect electric dipole is placed into a non-uniform electric field, the dipole can experience a net force.
\end{itemize}

%%%%%%%%%%%%%%%%%%%%%%%%%%%%%%%%%%%%%%%%%%%%%%%%%%%%%%%%%%%%%%%%%%%%%%%%%%%%%%%%%%%%%%%%%%%%

\pagebreak
\section{Answers}
\hspace{1cm}
\begin{spacing}{0.5}

\paragraph{1.1 $\rightarrow$} Coulomb
\paragraph{1.2 $\rightarrow$} $-1.602 \times 10^{-19}$ C
\paragraph{1.3 $\rightarrow$} $2.749 \times 10^{26}$ electrons

\paragraph{2.1 $\rightarrow$} $8.214 \times 10^{-8}$ N
\paragraph{2.2 $\rightarrow$} 0 N
\paragraph{2.3 $\rightarrow$} 
\paragraph{2.4 $\rightarrow$} In order to get two nuclei to fuse, one must launch them toward each other with a speed that can overcome the Coulomb repulsive force.

\paragraph{3.1 $\rightarrow$} Induction
\paragraph{3.2 $\rightarrow$} Conduction/Contact
\paragraph{3.3 $\rightarrow$} Friction

\paragraph{4.1 $\rightarrow$} $-\hat{x}$ direction
\paragraph{4.2 $\rightarrow$} $\vec{E} = \frac{kQz}{(z^2 + R^2)^{3/2}}\hat{z}$
\paragraph{4.3 $\rightarrow$} $\vec{E} \approx \frac{kQz}{R^3} \rightarrow \vec{F} = m \vec{a} \Rightarrow -\frac{kQz}{R^3} = m\frac{d^2z}{dt^2} \Rightarrow \frac{d^2z}{dt^2} + \frac{kQz}{mR^3} = 0 \Rightarrow z(t) = Asin(\omega t)$ where $\omega = \sqrt{\frac{kQ}{mR^3}}$
\paragraph{4.4 $\rightarrow$}
\paragraph{4.5 $\rightarrow$}

\paragraph{5.1 $\rightarrow$} $\vec{a} = 4.735 \times 10^{13}$ m/s$^2$
\paragraph{5.2 $\rightarrow$}
\paragraph{5.3 $\rightarrow$}
\paragraph{5.4 $\rightarrow$} D

\paragraph{6.1 $\rightarrow$} $-2kQ \hat{x}$
\paragraph{6.2 $\rightarrow$} 0 J
\paragraph{6.3 $\rightarrow$} 600 Nm
\paragraph{6.4 $\rightarrow$} A

\end{spacing}
\end{document}