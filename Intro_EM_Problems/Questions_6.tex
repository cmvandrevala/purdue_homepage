\title{Introductory E\&M Practice Problems}
\author{Cyrus Vandrevala}

\documentclass[11pt]{article}
\usepackage[margin=1.0in]{geometry}
\usepackage{setspace}

\begin{document}

\maketitle
\tableofcontents
\hspace{30mm}

\subsection*{Useful Constants}
Electron Mass = $9.11 \times 10^{-31}$ kg \\*
Proton Mass = $1.67 \times 10^{-27}$ kg \\*
Elementary Charge = $1.602 \times 10^{-19}$ C \\*
Coulomb's Constant = $8.99 \times 10^9$ Nm$^2$/C$^2$ \\*
Permittivity of Free Space = $8.85 \times 10^{-12}$ F/m \\*
Permeability of Free Space = $1.26 \times 10^{-6}$ m kg/s$^2$A$^2$ \\*

%%%%%%%%%%%%%%%%%%%%%%%%%%%%%%%%%%%%%%%%%%%%%%%%%%%%%%%%%%%%%%%%%%%%%%%%%%%%%%%%%%%%%%%%%%%%

\pagebreak
\section{Mirrors}

\subsection{Plane Mirror}
Our good friend Bob the college student has a date with a beautiful girl tonight!  Sadly, he is a scruffy guy, so he needs to clean up before his date.  He wants to go out to buy a mirror so he can comb his hair properly, but he needs to save as much money as possible for dinner at a super fancy restaurant.  How tall of a mirror should Bob buy if he is 6 feet tall and he wants to see his entire body all at once?
$\Rightarrow$ 3 ft

\subsection{Mirror Problem \#1}
A converging mirror has a focal length of $|f|$ = 10 cm.  I place an object with a height of 6 cm, 30 cm in front of the mirror.  

\begin{itemize}
\item[A)] What is the final image distance?
\item[B)] What is the final image height?
\item[C)] Is it in front of the mirror or behind the mirror?  
\item[D)] Is it real or virtual?
\item[E)] Is it upright or inverted?
\end{itemize}

\subsection{Mirror Problem \#2}
A diverging mirror has a focal length of $|f|$ = 15 cm.  I place an object 10 cm in front of the mirror.  

\begin{itemize}
\item[A)] What is the final image distance?
\item[B)] What is the final image height?
\item[C)] Is it in front of the mirror or behind the mirror?  
\item[D)] Is it real or virtual?
\item[E)] Is it upright or inverted?
\end{itemize}

\subsection{Mirror Problem \#3}
A converging mirror has a focal length of $|f|$ = 40 cm.  I place an object with height 5 cm, 10 cm in front of the mirror.  

\begin{itemize}
\item[A)] What is the final image distance?
\item[B)] What is the final image height?
\item[C)] Is it in front of the mirror or behind the mirror?  
\item[D)] Is it real or virtual?
\item[E)] Is it upright or inverted?
\end{itemize}

\subsection{Mirror Problem \#4}
A diverging mirror has a focal length of $|f|$ = 20 cm.  I place an object with height 12 cm, 30 cm in front of the mirror.  

\begin{itemize}
\item What is the final image distance?
\item What is the final image height?
\item Is it in front of the mirror or behind the mirror?  
\item Is it real or virtual?
\item Is it upright or inverted?
\end{itemize}
%%%%%%%%%%%%%%%%%%%%%%%%%%%%%%%%%%%%%%%%%%%%%%%%%%%%%%%%%%%%%%%%%%%%%%%%%%%%%%%%%%%%%%%%%%%%

\pagebreak
\section{Lenses}

\subsection{Lens Problem \#1}
A converging mirror has a focal length of |f| = 10 cm.  I place an object with a height of 6 cm, 30 cm in front of the mirror.  

\begin{itemize}
\item What is the final image distance?
\item What is the final image height?
\item Is it in front of the mirror or behind the mirror?  
\item Is it real or virtual?
\item Is it upright or inverted?
\end{itemize}

\subsection{Lens Problem \#2}
A diverging mirror has a focal length of |f| = 15 cm.  I place an object 10 cm in front of the mirror.  

\begin{itemize}
\item What is the final image distance?
\item What is the final image height?
\item Is it in front of the mirror or behind the mirror?  
\item Is it real or virtual?
\item Is it upright or inverted?
\end{itemize}

\subsection{Lens Problem \#3}
A converging mirror has a focal length of |f| = 40 cm.  I place an object with height 5 cm, 10 cm in front of the mirror.  

\begin{itemize}
\item What is the final image distance?
\item What is the final image height?
\item Is it in front of the mirror or behind the mirror?  
\item Is it real or virtual?
\item Is it upright or inverted?
\end{itemize}

\subsection{Lens Problem \#4}
A diverging mirror has a focal length of |f| = 20 cm.  I place an object with height 12 cm, 30 cm in front of the mirror.  

\begin{itemize}
\item What is the final image distance?
\item What is the final image height?
\item Is it in front of the mirror or behind the mirror?  
\item Is it real or virtual?
\item Is it upright or inverted?
\end{itemize}



%%%%%%%%%%%%%%%%%%%%%%%%%%%%%%%%%%%%%%%%%%%%%%%%%%%%%%%%%%%%%%%%%%%%%%%%%%%%%%%%%%%%%%%%%%%%


\pagebreak
\section{Interference Effects}

\subsection{Double Slit Experiment}
Consider an electric dipole with charges +Q and -Q located at x = +1 m and x = -1 m respectively.  Calculate the net electric field at x = 0 m.

\subsection{Potential Energy of an Electric Dipole}
A place a dipole with dipole moment p = 10 Cm [+z] in an electric field E = 100 N/C [-x].  What is the potential energy of the dipole?

\subsection{Maximum Potential Energy of an Electric Dipole}
Suppose I put a polar object with dipole moment 2 Cm in a uniform electric field of magnitude 300 N/C.  What is the maximum possible value for the potential energy of the system?

\subsection{Multiple Choice}
Which of the following statements is NOT true?

\begin{itemize}
	\item[A)] If a perfect electric dipole is placed into a uniform electric field, the dipole can experience a net force.
	\item[B)] A perfect electric dipole consists of two equal and opposite point charges separated by a small distance.
	\item[C)] If a perfect electric dipole is placed into a non-uniform electric field, the dipole can experience a net force.
	\item[D)] If a perfect electric dipole is placed into a uniform electric field, the dipole can experience a net torque.
	\item[E)] If a perfect electric dipole is placed into a non-uniform electric field, the dipole can experience a net force.
\end{itemize}

%%%%%%%%%%%%%%%%%%%%%%%%%%%%%%%%%%%%%%%%%%%%%%%%%%%%%%%%%%%%%%%%%%%%%%%%%%%%%%%%%%%%%%%%%%%%

\end{document}